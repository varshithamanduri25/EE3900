\documentclass[journal,12pt,twocolumn]{IEEEtran}
%
\usepackage{setspace}
\usepackage{gensymb}
\usepackage{xcolor}
\usepackage{caption}
%\usepackage{subcaption}
%\doublespacing
\singlespacing

%\usepackage{graphicx}
%\usepackage{amssymb}
%\usepackage{relsize}
\usepackage[cmex10]{amsmath}
\usepackage{mathtools}
%\usepackage{amsthm}
%\interdisplaylinepenalty=2500
%\savesymbol{iint}
%\usepackage{txfonts}
%\restoresymbol{TXF}{iint}
%\usepackage{wasysym}
\usepackage{hyperref}
\usepackage{amsthm}
\usepackage{mathrsfs}
\usepackage{txfonts}
\usepackage{stfloats}
\usepackage{cite}
\usepackage{cases}
\usepackage{subfig}
%\usepackage{xtab}
\usepackage{longtable}
\usepackage{multirow}
%\usepackage{algorithm}
%\usepackage{algpseudocode}
%\usepackage{enumerate}
\usepackage{enumitem}
\usepackage{mathtools}
%\usepackage{iithtlc}
%\usepackage[framemethod=tikz]{mdframed}
\usepackage{listings}


%\usepackage{stmaryrd}


%\usepackage{wasysym}
%\newcounter{MYtempeqncnt}
\DeclareMathOperator*{\Res}{Res}
%\renewcommand{\baselinestretch}{2}
\renewcommand\thesection{\arabic{section}}
\renewcommand\thesubsection{\thesection.\arabic{subsection}}
\renewcommand\thesubsubsection{\thesubsection.\arabic{subsubsection}}

\renewcommand\thesectiondis{\arabic{section}}
\renewcommand\thesubsectiondis{\thesectiondis.\arabic{subsection}}
\renewcommand\thesubsubsectiondis{\thesubsectiondis.\arabic{subsubsection}}

%\renewcommand{\labelenumi}{\textbf{\theenumi}}
%\renewcommand{\theenumi}{P.\arabic{enumi}}

% correct bad hyphenation here
\hyphenation{op-tical net-works semi-conduc-tor}

\lstset{
language=Python,
frame=single, 
breaklines=true,
columns=fullflexible
}



\begin{document}
%

\theoremstyle{definition}
\newtheorem{theorem}{Theorem}[section]
\newtheorem{problem}{Problem}
\newtheorem{proposition}{Proposition}[section]
\newtheorem{lemma}{Lemma}[section]
\newtheorem{corollary}[theorem]{Corollary}
\newtheorem{example}{Example}[section]
\newtheorem{definition}{Definition}[section]
%\newtheorem{algorithm}{Algorithm}[section]
%\newtheorem{cor}{Corollary}
\newcommand{\BEQA}{\begin{eqnarray}}
\newcommand{\EEQA}{\end{eqnarray}}
\newcommand{\define}{\stackrel{\triangle}{=}}

\bibliographystyle{IEEEtran}
%\bibliographystyle{ieeetr}

\providecommand{\nCr}[2]{\,^{#1}C_{#2}} % nCr
\providecommand{\nPr}[2]{\,^{#1}P_{#2}} % nPr
\providecommand{\mbf}{\mathbf}
\providecommand{\pr}[1]{\ensuremath{\Pr\left(#1\right)}}
\providecommand{\qfunc}[1]{\ensuremath{Q\left(#1\right)}}
\providecommand{\sbrak}[1]{\ensuremath{{}\left[#1\right]}}
\providecommand{\lsbrak}[1]{\ensuremath{{}\left[#1\right.}}
\providecommand{\rsbrak}[1]{\ensuremath{{}\left.#1\right]}}
\providecommand{\brak}[1]{\ensuremath{\left(#1\right)}}
\providecommand{\lbrak}[1]{\ensuremath{\left(#1\right.}}
\providecommand{\rbrak}[1]{\ensuremath{\left.#1\right)}}
\providecommand{\cbrak}[1]{\ensuremath{\left\{#1\right\}}}
\providecommand{\lcbrak}[1]{\ensuremath{\left\{#1\right.}}
\providecommand{\rcbrak}[1]{\ensuremath{\left.#1\right\}}}
\theoremstyle{remark}
\newtheorem{rem}{Remark}
\newcommand{\sgn}{\mathop{\mathrm{sgn}}}
\providecommand{\abs}[1]{\mathop{\left\vert#1\right\vert}}
\providecommand{\res}[1]{\Res\displaylimits_{#1}} 
\providecommand{\norm}[1]{\lVert#1\rVert}
\providecommand{\mtx}[1]{\mathbf{#1}}
\providecommand{\mean}[1]{E\mathop{\left[ #1 \right]}}
\providecommand{\fourier}{\overset{\mathcal{F}}{ \rightleftharpoons}}
\providecommand{\ztrans}{\overset{\mathcal{Z}}{ \rightleftharpoons}}

%\providecommand{\hilbert}{\overset{\mathcal{H}}{ \rightleftharpoons}}
\providecommand{\system}{\overset{\mathcal{H}}{ \longleftrightarrow}}
	%\newcommand{\solution}[2]{\textbf{Solution:}{#1}}
\newcommand{\solution}{\noindent \textbf{Solution: }}
\providecommand{\dec}[2]{\ensuremath{\overset{#1}{\underset{#2}{\gtrless}}}}
\numberwithin{equation}{section}
%\numberwithin{equation}{subsection}
%\numberwithin{problem}{subsection}
%\numberwithin{definition}{subsection}
\makeatletter
\@addtoreset{figure}{problem}
\makeatother

\let\StandardTheFigure\thefigure
%\renewcommand{\thefigure}{\theproblem.\arabic{figure}}
\renewcommand{\thefigure}{\theproblem}


%\numberwithin{figure}{subsection}

\def\putbox#1#2#3{\makebox[0in][l]{\makebox[#1][l]{}\raisebox{\baselineskip}[0in][0in]{\raisebox{#2}[0in][0in]{#3}}}}
     \def\rightbox#1{\makebox[0in][r]{#1}}
     \def\centbox#1{\makebox[0in]{#1}}
     \def\topbox#1{\raisebox{-\baselineskip}[0in][0in]{#1}}
     \def\midbox#1{\raisebox{-0.5\baselineskip}[0in][0in]{#1}}

\vspace{3cm}

\title{ 
%\logo{
Digital Signal Processing\\Assignment 1
%}
%	\logo{Octave for Math Computing }
}
%\title{
%	\logo{Matrix Analysis through Octave}{\begin{center}\includegraphics[scale=.24]{tlc}\end{center}}{}{HAMDSP}
%}


% paper title
% can use linebreaks \\ within to get better formatting as desired
%\title{Matrix Analysis through Octave}
%
%
% author names and IEEE memberships
% note positions of commas and nonbreaking spaces ( ~ ) LaTeX will not break
% a structure at a ~ so this keeps an author's name from being broken across
% two lines.
% use \thanks{} to gain access to the first footnote area
% a separate \thanks must be used for each paragraph as LaTeX2e's \thanks
% was not built to handle multiple paragraphs
%

\author{ Sri Varshitha Manduri %<-this  stops a space
\thanks{}% <-this % stops a space
%\thanks{J. Doe and J. Doe are with Anonymous University.}% <-this % stops a space
%\thanks{Manuscript received April 19, 2005; revised January 11, 2007.}}
}
% note the % following the last \IEEEmembership and also \thanks - 
% these prevent an unwanted space from occurring between the last author name
% and the end of the author line. i.e., if you had this:
% 
% \author{....lastname \thanks{...} \thanks{...} }
%                     ^------------^------------^----Do not want these spaces!
%
% a space would be appended to the last name and could cause every name on that
% line to be shifted left slightly. This is one of those "LaTeX things". For
% instance, "\textbf{A} \textbf{B}" will typeset as "A B" not "AB". To get
% "AB" then you have to do: "\textbf{A}\textbf{B}"
% \thanks is no different in this regard, so shield the last } of each \thanks
% that ends a line with a % and do not let a space in before the next \thanks.
% Spaces after \IEEEmembership other than the last one are OK (and needed) as
% you are supposed to have spaces between the names. For what it is worth,
% this is a minor point as most people would not even notice if the said evil
% space somehow managed to creep in.



% The paper headers
%\markboth{Journal of \LaTeX\ Class Files,~Vol.~6, No.~1, January~2007}%
%{Shell \MakeLowercase{\textit{et al.}}: Bare Demo of IEEEtran.cls for Journals}
% The only time the second header will appear is for the odd numbered pages
% after the title page when using the twoside option.
% 
% *** Note that you probably will NOT want to include the author's ***
% *** name in the headers of peer review papers.                   ***
% You can use \ifCLASSOPTIONpeerreview for conditional compilation here if
% you desire.




% If you want to put a publisher's ID mark on the page you can do it like
% this:
%\IEEEpubid{0000--0000/00\$00.00~\copyright~2007 IEEE}
% Remember, if you use this you must call \IEEEpubidadjcol in the second
% column for its text to clear the IEEEpubid mark.



% make the title area
\maketitle

%\newpage

\tableofcontents

%\renewcommand{\thefigure}{\thesection.\theenumi}
%\renewcommand{\thetable}{\thesection.\theenumi}

\renewcommand{\thefigure}{\theenumi}
\renewcommand{\thetable}{\theenumi}

%\renewcommand{\theequation}{\thesection}


\bigskip

\begin{abstract}
This manual provides a simple introduction to digital signal processing.
\end{abstract}
\section{Software Installation}
Run the following commands
\begin{lstlisting}
sudo apt-get update
sudo apt-get install libffi-dev libsndfile1 python3-scipy  python3-numpy python3-matplotlib 
sudo pip install cffi pysoundfile 
\end{lstlisting}
\section{Digital Filter}
\begin{enumerate}[label=\thesection.\arabic*
,ref=\thesection.\theenumi]
\item
\label{prob:input}
Download the sound file from  
\begin{lstlisting}
wget https://raw.githubusercontent.com/gadepall/ 
EE1310/master/filter/codes/Sound_Noise.wav
\end{lstlisting}
%\href{http://tlc.iith.ac.in/img/sound/Sound_Noise.wav}{\url{http://tlc.iith.ac.in/img/sound/Sound_Noise.wav}}  
%in the link given below.
%\linebreak
\item
\label{prob:spectrogram}
You will find a spectrogram at \href{https://academo.org/demos/spectrum-analyzer}{\url{https://academo.org/demos/spectrum-analyzer}}. 
%\end{problem}
%%
%
%%\onecolumn
%%\input{./figs/fir}
%\begin{problem}
Upload the sound file that you downloaded in Problem \ref{prob:input} in the spectrogram  and play.  Observe the spectrogram. What do you find?
\\
%
\solution\\
\includegraphics[width=\columnwidth]{figs/sound_noise}\\
There are a lot of yellow lines between 440 Hz to 5.1 KHz.  These represent the synthesizer key tones. Also, the key strokes
are audible along with background noise.
% By observing spectrogram, it clearly shows that tonal frequency is under 4kHz. And above 4kHz only noise is present.
\item
\label{prob:output}
Write the python code for removal of out of band noise and execute the code.
\\
\solution
\lstinputlisting{./codes/Cancel_noise.py}
%\begin{figure}[h]
%\centering
%\includegraphics[width=\columnwidth]{enc_block_diag.png}
%\caption{}
%\label{fig:convolution encoder}
%\end{figure}
%\input{block_enc}
\item
The output of the python script in Problem \ref{prob:output} is the audio file Sound\_With\_ReducedNoise.wav. Play the file in the spectrogram in Problem \ref{prob:spectrogram}. What do you observe?
\\
\solution\\
\includegraphics[width=\columnwidth]{figs/sound_reduced_noise}\\
The key strokes as well as background noise is subdued in the audio.  Also,  the signal is blank for frequencies above 5.1 kHz.

\end{enumerate}
\section{Difference Equation}
\begin{enumerate}[label=\thesection.\arabic*,ref=\thesection.\theenumi]
\item Let
	\label{def:xn}
\begin{equation}
x(n) = \cbrak{\underset{\uparrow}{1},2,3,4,2,1}
\end{equation}
Sketch $x(n)$.
\\
\solution
\lstinputlisting{./codes/xn.py}
\begin{figure}[!ht]
\begin{center}
\includegraphics[width=\columnwidth]{figs/xn.png}
\end{center}
\captionof{figure}{}
\label{fig:xn}	
\end{figure}
\item Let
\begin{multline}
\label{eq:iir_filter}
y(n) + \frac{1}{2}y(n-1) = x(n) + x(n-2), 
\\
 y(n) = 0, n < 0
\end{multline}
Sketch $y(n)$.  
\\
\solution The following code yields Fig. \ref{fig:xnyn}.
\lstinputlisting{codes/xnyn.py}
\begin{figure}[!ht]
\begin{center}
\includegraphics[width=\columnwidth]{figs/xnyn.pdf}
\end{center}
\captionof{figure}{}
\label{fig:xnyn}	
\end{figure}
\item Repeat the above exercise using a C code.
\solution
\lstinputlisting{codes/3.3.c}
For plotting:
\lstinputlisting{codes/3.3.py}
\end{enumerate}
\section{$Z$-transform}
\begin{enumerate}[label=\thesection.\arabic*]
\item The $Z$-transform of $x(n)$ is defined as
%
\begin{equation}
\label{eq:z_trans}
X(z)={\mathcal {Z}}\{x(n)\}=\sum _{n=-\infty }^{\infty }x(n)z^{-n}
\end{equation}
%
Show that
\begin{equation}
\label{eq:shift1}
{\mathcal {Z}}\{x(n-1)\} = z^{-1}X(z)
\end{equation}
and find
\begin{equation}
	{\mathcal {Z}}\{x(n-k)\} 
\end{equation}
\solution From \eqref{eq:z_trans},
\begin{align}
{\mathcal {Z}}\{x(n-k)\} &=\sum _{n=-\infty }^{\infty }x(n-1)z^{-n}
\\
&=\sum _{n=-\infty }^{\infty }x(n)z^{-n-1} = z^{-1}\sum _{n=-\infty }^{\infty }x(n)z^{-n}
\end{align}
resulting in \eqref{eq:shift1}. Similarly, it can be shown that
%
\begin{equation}
\label{eq:z_trans_shift}
	{\mathcal {Z}}\{x(n-k)\} = z^{-k}X(z)
\end{equation}
\item Obtain $X(z)$ for $x(n)$ defined in problem 
	\ref{def:xn}.
\\

\solution 
\begin{equation}
x(n) = \cbrak{1,2,3,4,2,1}
\end{equation}

\begin{align}
X(z)&=\sum _{n=-\infty }^{\infty }x(n)z^{-n}
\\& =z^{-1}+2z^{-2}+3z^{-3}+4z^{-4}+2z^{-2}+1z^{-1}
\end{align}
\item Find
%
\begin{equation}
H(z) = \frac{Y(z)}{X(z)}
\end{equation}
%
from  \eqref{eq:iir_filter} assuming that the $Z$-transform is a linear operation.
\\
\solution  Applying \eqref{eq:z_trans_shift} in \eqref{eq:iir_filter},
\begin{align}
Y(z) + \frac{1}{2}z^{-1}Y(z) &= X(z)+z^{-2}X(z)
\\
\implies \frac{Y(z)}{X(z)} &= \frac{1 + z^{-2}}{1 + \frac{1}{2}z^{-1}}
\label{eq:freq_resp}
\end{align}
%
\item Find the Z transform of 
\begin{equation}
\delta(n)
=
\begin{cases}
1 & n = 0
\\
0 & \text{otherwise}
\end{cases}
\end{equation}
and show that the $Z$-transform of
\begin{equation}
\label{eq:unit_step}
u(n)
=
\begin{cases}
1 & n \ge 0
\\
0 & \text{otherwise}
\end{cases}
\end{equation}
is
\begin{equation}
U(z) = \frac{1}{1-z^{-1}}, \quad \abs{z} > 1
\end{equation}
\solution It is easy to show that
\begin{equation}
\delta(n) \ztrans 1
\end{equation}
and from \eqref{eq:unit_step},
\begin{align}
U(z) &= \sum _{n= 0}^{\infty}z^{-n}
\\
&=\frac{1}{1-z^{-1}}, \quad \abs{z} > 1
\end{align}

%
\item Show that 
\begin{equation}
\label{eq:anun}
a^nu(n) \ztrans \frac{1}{1-az^{-1}} \quad \abs{z} > \abs{a}
\end{equation}
\solution 
\begin{equation}
Since ,
\label{eq:unit_step}
u(n)
=
\begin{cases}
1 & n \ge 0
\\
0 & \text{otherwise}
\end{cases}
\end{equation}
\begin{equation}
a^nu(n) &= \cbrak {1,a^1,a^2..}
\end{equation} 
Z transform of this,
\begin{align}
&= \sum _{n=-\infty }^{\infty }a^nz^{-n}
\\&= 1+az^{-1}+a^2z^{-2}+\dots
\\&a^nu(n) \ztrans \frac{1}{1-az^{-1}}
 (\abs{z} > \abs{a} )
\end{align}
using the formula for the sum of an infinite geometric progression.
%
\item 
Let
\begin{equation}
H\brak{e^{\j \omega}} = H\brak{z = e^{\j \omega}}.
\end{equation}
Plot $\abs{H\brak{e^{\j \omega}}}$.  Is it periodic? If so, find the period. $H(e^{\j \omega})$ is
known as the {\em Discret Time Fourier Transform} (DTFT) of $h(n)$.
\\
\solution The following code plots Fig. \ref{fig:dtft}.
\lstinputlisting{codes/dtft.py}
\begin{figure}[!ht]
\centering
\includegraphics[width=\columnwidth]{figs/dtft}
\caption{$\abs{H\brak{e^{\j\omega}}}$}
\label{fig:dtft}
\end{figure}
Proving its periodic,
\begin{equation}
	H(z) = \frac{Y(z)}{X(z)}= \frac{1 + z^{-2}}{1 + \frac{1}{2}z^{-1}}	
\end{equation}
for 
\begin{align}
        z &= e^{jw}
	\\H(z)&=H(e^{jw}) =  \frac{1 + e^{-2jw}}{1 + \frac{1}{2}e^{-jw}}
	\\&= \frac{1 + e^{2jw}}{e^{2jw} + \frac{1}{2}e^{jw}}
	\\ \implies \abs{H(e^{jw})} &= \abs{\frac{1 + e^{2jw}}{e^{2jw} + \frac{1}{2}e^{jw}} }
	\\&= \frac{\abs{1 + e^{2jw}}}{\abs{e^{2jw} + \frac{1}{2}e^{jw}}}
	\\&= \frac{\abs{1+cos2w+2jsinw}*2}{\abs{e^{jw} + 1}}
	\\&= \frac{\abs{cosw+jsinw}*4cosw}{\abs{2cosw+1+2jsinw}}
	\\&= \frac{4cosw}{\sqrt{(2cosw+1)^{2}+4sin^{2}w}}
	\\&= \frac{4cosw}{\sqrt{5+4cosw}}
\end{align}
Therefore 
\begin{equation}
	H(e^{jw}) = \frac{4cosw}{\sqrt{5+4cosw}}
\end{equation}
Consider 
\begin{equation}
	\label{eq:func_period}
	f\brak{x}=\frac{4cosx}{\sqrt{5+4cosx}}
\end{equation}
then
\begin{equation}
	\label{eq:func_period_new}
	f\brak{x+t}=\frac{4cos(x+t)}{\sqrt{5+4cos(x+t)}}
\end{equation}
\begin{align}
	f\brak{x+t}=\frac{4cos(x+t)}{\sqrt{5+4cos(x+t)}}
	\\=\frac{4(cosxcost-sinxsint)}{\sqrt{5+4(cosxcost-sinxsint)}}
\end{align}
By comparing \eqref{eq:func_period} and \eqref{eq:func_period_new}, we get\\
cost=1 and sint=0
\\This is true for $t=2k\pi$. This implies that the principal period of this function is $2\pi$.
\item Express $h(n)$ in terms of $H\brak{e^{\j \omega}}$.
\begin{align}
	H\brak{e^{j\omega}}=\sum _{n= -\infty}^{\infty} h\brak{n} e^{-jn\omega }
	\\\implies\brak{e^{j\omega}}*e^{jk\omega} &= \sum _{n= -\infty}^{\infty} h\brak{n} e^{-jn\omega} e^{jk\omega}
	\\\implies\int_{-\pi}^{\pi} H\brak{e^{j\omega}}*e^{jk\omega}d\omega &= \int_{-\pi}^{\pi}\sum _{n= -\infty}^{\infty} h\brak{n} e^{-jn\omega} e^{jk\omega}
	\\\implies\int_{-\pi}^{\pi} H\brak{e^{j\omega}}*e^{jk\omega}d\omega &= \sum _{n= -\infty}^{\infty} h\brak{n}\int_{-\pi}^{\pi} e^{-jn\omega} e^{jk\omega}
\end{align}

NOTE:
We know that,
\begin{equation}
	\int_{-\pi}^{\pi} e^{-jn\omega} e^{jk\omega} 
	=
	\begin{cases}
		2\pi & n = k
		\\
		0 & \text{otherwise}
		\end{cases}
\end{equation}	


Hence,
\begin{align}
	\int_{-\pi}^{\pi} H\brak{e^{j\omega}}*e^{jk\omega}d\omega = \int_{-\pi}^{\pi}\sum _{n= -\infty}^{\infty} h\brak{n} e^{-jn\omega} e^{jk\omega}
	\\\implies \int_{-\pi}^{\pi} H\brak{e^{j\omega}}*e^{jk\omega}d\omega = 2\pi h\brak{n}
	\\\implies \frac{1}{2\pi} \int_{-\pi}^{\pi} H\brak{e^{j\omega}}*e^{jk\omega}d\omega = h\brak{n}
\end{align}
 
Therefore,\\
\begin{align}
	h(n) &= \frac{1}{2\pi}\int_{-\pi}^{\pi}H\brak{e^{j\omega}}e^{j\omega n}d\omega
\end{align}
\end{enumerate}

\section{Impulse Response}
\begin{enumerate}[label=\thesection.\arabic*]
	\item Using long division, 
find
		\begin{align}
			h(n), \quad n < 5
		\end{align}
		for H(z) in 
		\eqref{eq:freq_resp}.
  \\\solution\\
	\begin{equation}
		H(z) = \frac{1 + z^{-2}}{1 + \frac{1}{2} z^{-1}}
	\end{equation}
	Let $z^{-1} = x$,then, by polynomial long division we get
	
\begin{align}
	\implies (1+z^{-2})= (\frac{1}{2} z^{-1}+1)(2z^{-1}-4) + 5
	\\\implies \frac{(1+z^{-2})}{\frac{1}{2} z^{-1}+1}= (2z^{-1}-4) + \frac{5}{\frac{1}{2} z^{-1}+1}
	\\\implies H\brak{z}=(2z^{-1}-4) + \frac{5}{\frac{1}{2} z^{-1}+1}
\end{align}

Now, consider$ \frac{5}{\frac{1}{2} z^{-1}+1}$
\\The denominator $\frac{1}{2} z^{-1}+1$ can be expressed as sum of an infinite geometric progression, which as its first term equal to 1 and common ratio $\frac{-1}{2}z^{-1}$
\\Therefore, we can write $ \frac{5}{\frac{1}{2} z^{-1}+1}$ as 5\brak{1+\brak{\frac{-1}{2}z^{-1}}+\brak{\frac{-1}{2}z^{-1}}^{2}+\brak{\frac{-1}{2}z^{-1}}^{3}+\brak{\frac{-1}{2}z^{-1}}^{4}+\dots}	
\\Therefore, H(z) can be given by,
\\\begin{equation}
	H(z)= (2z^{-1}-4) + \frac{5}{\frac{1}{2} z^{-1}+1}
\end{equation}
\begin{align}
	\\= 2z^{-1} - 4 + 5 + \frac{-5}{2}z^{-1} + \frac{5}{4}z^{-2} + \frac{-5}{8}z^{-3} + \frac{5}{16}z^{-4} + \dots
	\\\implies H(z)= 1z^{0} + \frac{-1}{2}z^{-1} +\frac{5}{4}z^{-2} + \frac{-5}{8}z^{-3} +\frac{5}{16}z^{-4} +\dots
\end{align}
Comparing the above expression to \eqref{eq:z_trans} we get h(n) for n$<$5 as, \\
\begin{align} 
	h(0) &= 1
	\\h(1) &= \frac{-1}{2}
	\\h(2) &= \frac{5}{4}
	\\h(3) &= \frac{-5}{8}
	\\h(4) &= \frac{5}{16}
\end{align}

\item \label{prob:impulse_resp}
Find an expression for $h(n)$ using $H(z)$, given that 
%in Problem \ref{eq:ztransab} and \eqref{eq:anun}, given that
\begin{equation}
\label{eq:impulse_resp}
h(n) \ztrans H(z)
\end{equation}
and there is a one to one relationship between $h(n)$ and $H(z)$. $h(n)$ is known as the {\em impulse response} of the
system defined by \eqref{eq:iir_filter}.
\\
\solution From \eqref{eq:freq_resp},
\begin{align}
H(z) &= \frac{1}{1 + \frac{1}{2}z^{-1}} + \frac{ z^{-2}}{1 + \frac{1}{2}z^{-1}}
\\
\implies h(n) &= \brak{-\frac{1}{2}}^{n}u(n) + \brak{-\frac{1}{2}}^{n-2}u(n-2)
\label{eq:hn_exp}
\end{align}
using \eqref{eq:anun} and \eqref{eq:z_trans_shift}.
\item Sketch $h(n)$. Is it bounded? Justify theoretically.
\\
\solution The following code plots Fig. \ref{fig:hn}.
\lstinputlisting{codes/hn.py}
\begin{figure}[!ht]
\centering
\includegraphics[width=\columnwidth]{./figs/hn}
\caption{$h(n)$ as the inverse of $H(z)$}
\label{fig:hn}
\end{figure}
From \eqref{eq:hn_exp} we know that\\
\begin{equation}
	h(n) = \brak{-\frac{1}{2}}^{n}u(n) + \brak{-\frac{1}{2}}^{n-2}u(n-2)
\end{equation}
Implies we can write that\\
\begin{align}
	h\brak{n} &= \begin{cases}
					 0 &, n < 0\\
					 \brak{\frac{-1}{2}}^n &, 0\leq n<2\\
					 5\brak{\frac{-1}{2}}^n &, n \geq 2
				  \end{cases}\label{eq:hn_theoritical_def}
\end{align}
A sequence is said to be bounded when 
\begin{align}
	\abs{x_n} \leq M , \forall n \in \mathcal{N}\label{def:bounded_seq_def}
\end{align}  

Now consider \eqref{eq:hn_theoritical_def},\\
For n$<$0,
\begin{equation}
	\abs{h\brak{n}} \leq 0
\end{equation}
For $0 \leq n <$ 2,
\begin{align}
	\abs{h\brak{n}} = (\frac{1}{2})^n\\
	\implies \abs{h\brak{n}} \leq 1
\end{align}
For $n\geq 2$,
\begin{align}
	\abs{h\brak{n}} = 5(\frac{1}{2})^n\\
	\implies \abs{h\brak{n}} \leq 5
\end{align}

From above we can say that,
      \begin{align}
        M &= max\cbrak{0,1,5} \\
          &= 5\label{eq:hn_bounded_proof}
      \end{align}

Therefore since $M$ exists and is a real value, we can say that h\brak{n} is bounded.
\item Convergent? Justify using the ratio test.
\solution\\
We can check if a sequence os convergent by ratio test. From the defination of ratio test we can say that a sequence is convergent if 
\begin{equation}
	\lim_{n \rightarrow \infty}\abs{\frac{x_{n+1}}{x_n}} < 1
\end{equation}
Here, applying ratio test on \eqref{eq:hn_exp} is same as applying ratio test on \eqref{eq:hn_theoritical_def}
\begin{align}
	\lim_{n \rightarrow \infty}\abs{\frac{h(n+1)}{h(n)}} = \lim_{n \rightarrow \infty}\abs{\frac{5(\frac{-1}{2})^{n+1}}{5(\frac{-1}{2})^{n}}}
	\\= \abs{\frac{-1}{2}}
	\\= \frac{1}{2}\label{eq:convergent_proof}
\end{align}
From \eqref{eq:convergent_proof} we can clearly say that the sequence h\brak{n} is convergent.

\item The system with $h(n)$ is defined to be stable if
\begin{equation}
\sum_{n=-\infty}^{\infty}h(n) < \infty
\end{equation}
Is the system defined by \eqref{eq:iir_filter} stable for the impulse response in \eqref{eq:impulse_resp}?
\\\solution\\
From \eqref{eq:hn_exp} we know that,
\begin{equation}
	h(n) = \brak{-\frac{1}{2}}^{n}u(n) + \brak{-\frac{1}{2}}^{n-2}u(n-2)
\end{equation}
Given that, a system is stable when 
\begin{equation}
	\sum_{n=-\infty}^{\infty}h(n) < \infty
\end{equation}
\begin{align}
	\implies \sum_{n=-\infty}^{\infty}h(n) = \sum_{n=-\infty}^{\infty} (\brak{-\frac{1}{2}}^{n}u(n) + \brak{-\frac{1}{2}}^{n-2}u(n-2))
	\\ = 2*\sum_{n=-\infty}^{\infty} \brak{-\frac{1}{2}}^{n}u(n)
\end{align}
\begin{align}
	\implies \sum_{n=-\infty}^{\infty}h(n) = 2*(\frac{1}{1+\frac{1}{2}})
	\\= \frac{4}{3} < \infty
\end{align}
Hence, the system is stable.
\item Verify the above result using a python code.
\includegraphics[width=\columnwidth]{figs/5.6.png}
\item 
Compute and sketch $h(n)$ using 
\begin{equation}
\label{eq:iir_filter_h}
h(n) + \frac{1}{2}h(n-1) = \delta(n) + \delta(n-2), 
\end{equation}
%
This is the definition of $h(n)$.
\\
\solution The following code plots Fig. \ref{fig:hndef}. Note that this is the same as Fig. 
\ref{fig:hn}. 
%
\lstinputlisting{codes/hndef.py}
\begin{figure}[!ht]
\centering
\includegraphics[width=\columnwidth]{./figs/hndef}
\caption{$h(n)$ from the definition}
\label{fig:hndef}
\end{figure}
%
\item Compute 
%
\begin{equation}
\label{eq:convolution}
y(n) = x(n)*h(n) = \sum_{n=-\infty}^{\infty}x(k)h(n-k)
\end{equation}
%
Comment. The operation in \eqref{eq:convolution} is known as
{\em convolution}.
%
\\
\solution The following code plots Fig. \ref{fig:ynconv}. Note that this is the same as 
$y(n)$ in  Fig. 
\ref{fig:xnyn}. 
%
\lstinputlisting{codes/ynconv.py}
\begin{figure}[!ht]
\centering
\includegraphics[width=\columnwidth]{./figs/ynconv}
\caption{$y(n)$ from the definition of convolution}
\label{fig:ynconv}
\end{figure}
\item Express the above convolution using a Teoplitz matrix.
\\\solution\\

We know that from, \eqref{eq:convolution},
\begin{equation}
	y(n) = x(n)*h(n) = \sum_{k=-\infty}^{\infty}x(k)h(n-k)
\end{equation}
This can also be wrtten as a matrix-vector multiplication given by the expression,
\begin{equation}
	\label{eq:conv_matrix_vec_mult}
	y = T\brak{h}*x
\end{equation}
In the equation \eqref{eq:conv_matrix_vec_mult}, $T\brak{h}$ is a Teoplitz matrix.
\\ The equation \eqref{eq:conv_matrix_vec_mult} can be expanded as,
\begin{align}
	\mtx{y} &= \mtx{x} \circledast \mtx{h}\\
	\mtx{y} &= 
	\begin{pmatrix}
		h_1 & 0 & . & . & . & 0 \\
		h_2 & h_1 & . & . & . & 0 \\
		h_3 & h_2 & h_1 & . & . & 0 \\
		. & . & . & . & . & . \\
		h_{n-1} & h_{n-2} & h_{n-3} & . & . & 0\\
		h_{n} & h_{n-1} & h_{n-2} & . & . & h_1\\
		0 & h_{n} & h_{n-1} & h_{n-2} & . & h_2\\
		. & . & . & . & . & . \\
		0 & . & . & . & 0 & h_{n-1} \\
		0 & . & . & . & 0 & h_n \\
	\end{pmatrix}
	\begin{pmatrix}
		x_1 \\ x_2 \\ .\\.\\. \\ x_n
	\end{pmatrix}
\end{align}
\item Show that
\begin{equation}
y(n) =  \sum_{n=-\infty}^{\infty}x(n-k)h(k)
\end{equation}
\\\solution\\
From the defination of convolution given in \eqref{eq:convolution},we know that,
\begin{equation}
	y(n) = x(n)*h(n) = \sum_{k=-\infty}^{\infty}x(k)h(n-k)
\end{equation}
\begin{align}
	\implies y\brak{n} = \sum_{k=-\infty}^{\infty}x(k)h(n-k)
\end{align}
Replace $k$ with $n-k$. Now since n varies from n=$-\infty$ to n=$\infty$, n-k will also vary from $-\infty$ to $\infty$. Therefore, we get,
\begin{align}
	y\brak{n} = \sum_{n-k=-\infty}^{\infty}x(n-k)h(k)	
	\\=\sum_{k=-\infty}^{\infty}x(n-k)h(k)
\end{align}
\end{enumerate}

%
\section{DFT and FFT}
\begin{enumerate}[label=\thesection.\arabic*]
\item
Compute
\begin{equation}
X(k) \define \sum _{n=0}^{N-1}x(n) e^{-\j2\pi kn/N}, \quad k = 0,1,\dots, N-1
\end{equation}
and $H(k)$ using $h(n)$.
\\\solution\\
From \ref{def:xn}, we know that ,
\begin{equation}
	x(n) = \cbrak{\underset{\uparrow}{1},2,3,4,2,1}
\end{equation}
%  Let, $e^{-j2\pi k} = \omega$. Then,
% \begin{equation}
% 	X(k) = \sum _{n=0}^{N-1}x(n) \omega^{n/N}
% \end{equation}
Here, let, $\omega=e^{-j2\pi k}$. Then,
\begin{align}
	X(k)=1+2\omega^{\frac{1}{5}}+3\omega^{\frac{2}{5}}+4\omega^{\frac{3}{5}}+2\omega^{\frac{4}{5}}+\omega
	\label{eq:Xkdef}
\end{align}

Similarly, we know from \eqref{eq:hn_theoritical_def},
\begin{align}
	h\brak{n} &= \begin{cases}
					 0 &, n < 0\\
					 \brak{\frac{-1}{2}}^n &, 0\leq n<2\\
					 5\brak{\frac{-1}{2}}^n &, n \geq 2
				  \end{cases}
\end{align}
Now,again let, $\omega=e^{-j2\pi k}$. Then,
\begin{align}
	H(k)=1+\frac{-1}{2}\omega^{\frac{1}{5}}+\frac{5}{4}\omega^{\frac{2}{5}}+\frac{-5}{8}\omega^{\frac{3}{5}}+\frac{5}{16}\omega^{\frac{4}{5}}+\frac{-5}{32}\omega
	\label{eq:Hkdef}
\end{align}

\item Compute 
\begin{equation}
Y(k) = X(k)H(k)
\end{equation}
\\\solution\\Now, from \eqref{eq:Xkdef} and \eqref{eq:Hkdef}, we know $X(k) and H(k)$. Now, given that,
\begin{equation}
	Y\brak{k}=X\brak{k}*H\brak{k}
\end{equation}
%\polyadd\PolynomX{1+2\omega^{\frac{1}{5}}+3\omega^{\frac{2}{5}}+4\omega^{\frac{3}{5}}+2\omega^{\frac{4}{5}}+\omega}{0}
\begin{multline}
	Y\brak{k}= (1+2\omega^{\frac{1}{5}}+3\omega^{\frac{2}{5}}+4\omega^{\frac{3}{5}}+2\omega^{\frac{4}{5}}+\omega)* \\(1+\frac{-1}{2}\omega^{\frac{1}{5}}+\frac{5}{4}\omega^{\frac{2}{5}}+\frac{-5}{8}\omega^{\frac{3}{5}}+\frac{5}{16}\omega^{\frac{4}{5}}+\frac{-5}{32}\omega)
\end{multline}
\begin{multline}
	Y\brak{k}= 1+\frac{3}{2}\omega^{\frac{1}{5}}+\frac{13}{4}\omega^{\frac{2}{5}}+\frac{35}{8}\omega^{\frac{3}{5}}+\frac{45}{16}\omega^{\frac{4}{5}}\\\frac{115}{32}\omega^{\frac{5}{5}}+\frac{1}{8}\omega^{\frac{6}{5}}+\frac{25}{32}\omega^{\frac{7}{5}}-\frac{5}{8}\omega^{\frac{8}{5}}\\-\frac{5}{32}\omega^{5} 
\end{multline}
where, $\omega=e^{-j2k\pi}$
\item Compute
\begin{equation}
 y\brak{n}={\frac {1}{N}}\sum _{k=0}^{N-1}Y\brak{k}\cdot e^{\j 2\pi kn/N},\quad n = 0,1,\dots, N-1
\end{equation}
\\
\solution The following code plots Fig. \ref{fig:ynconv}. Note that this is the same as 
$y(n)$ in  Fig. 
\ref{fig:xnyn}. 
%
\lstinputlisting{codes/yndft.py}
\begin{figure}[!ht]
\centering
\includegraphics[width=\columnwidth]{./figs/yndft}
\caption{$y(n)$ from the DFT}
\label{fig:yndft}
\end{figure}

\item Repeat the previous exercise by computing $X(k), H(k)$ and $y(n)$ through FFT and 
IFFT.
\item Wherever possible, express all the above equations as matrix equations.
\item Verify the above equations by generating the DFT matrix in python.
\end{enumerate}
%
\section{Exercises}

Answer the following questions by looking at the python code in Problem \ref{prob:output}.
\begin{enumerate}[label=\thesection.\arabic*]
\item
The command
\begin{lstlisting}
	output_signal = signal.lfilter(b, a, input_signal)
	\end{lstlisting}
in Problem \ref{prob:output} is executed through the following difference equation
\begin{equation}
\label{eq:iir_filter_gen}
 \sum _{m=0}^{M}a\brak{m}y\brak{n-m}=\sum _{k=0}^{N}b\brak{k}x\brak{n-k}
\end{equation}
%
where the input signal is $x(n)$ and the output signal is $y(n)$ with initial values all 0. Replace
\textbf{signal.filtfilt} with your own routine and verify.
%
\item Repeat all the exercises in the previous sections for the above $a$ and $b$.

\item What is the sampling frequency of the input signal?
\\
\solution
Sampling frequency(fs)=44.1kHZ.
\item
What is type, order and  cutoff-frequency of the above butterworth filter
\\
\solution
The given butterworth filter is low pass with order=2 and cutoff-frequency=4kHz.
%
\item
Modifying the code with different input parameters and to get the best possible output.
%
\end{enumerate}

\end{document}

